\section{\Large PROBLEM SET 8}

\subsection{Problem 1 - The time update should be complete and verified at this stage. We need to include the incoming measurements. Search in literature or derive, define, and code a sensitivity matrix H which provides your state at step k+1 from your measurement vector at step k+1 (i.e., at the same current time, propagation has already been done).}

The sensitivity matrix is defined in Equation \ref{eq:general_sens_matrix} as being the Jacobian of the measurement model, $\vec{h}$, taken with respect to the state variables.

\begin{equation} \label{eq:general_sens_matrix}
    \boldsymbol{H} = \frac{\partial \vec{h}}{\partial \vec{x}}
\end{equation}

If the measurement model yields $N$ unit vectors of known direction in the inertial frame (i.e. star direction, sun direction, etc.) represented by vectors, $\vec{v}_i$, then it follows the form shown in Equation \ref{eq:attitude_meas_model}.

\begin{equation} \label{eq:attitude_meas_model}
    \vec{h}_{\alpha} = \begin{bmatrix}
        \boldsymbol{R}_{t+1 \vert t} \vec{v}_1 \\
        \boldsymbol{R}_{t+1 \vert t} \vec{v}_2 \\
        \vdots \\
        \boldsymbol{R}_{t+1 \vert t} \vec{v}_N
    \end{bmatrix} = \begin{bmatrix}
        \vec{h}_1 \\
        \vec{h}_2 \\
        \vdots \\
        \vec{h}_N
    \end{bmatrix}
\end{equation}

Therefore, the sensitivity matrix that corresponds to this measurement type is defined in Equation \ref{eq:attitude_sens_matrix}.

\begin{equation} \label{eq:attitude_sens_matrix}
    \boldsymbol{H}_{\alpha} = \begin{bmatrix}
        \boldsymbol{ \left[ h_1 \times \right] } & \boldsymbol{ \left[ h_2 \times \right] } & \dots & \boldsymbol{ \left[ h_N \times \right] }
    \end{bmatrix}^T
\end{equation}

In the state space that has been chosen for this Kalman Filter, however, there is also a component of the measurement and the component of the state that describe the angular velocity of the spacecraft about its principal axes. Tor this component of the measurement and state vector, the sensitivity matrix is simply identity, which is seen in Equations \ref{eq:vel_meas_model} and \ref{eq:vel_sens_matrix}.

\begin{equation} \label{eq:vel_meas_model}
    \vec{h}_{\omega} = \begin{bmatrix}
        \omega_x \\
        \omega_y \\
        \omega_z
    \end{bmatrix}
\end{equation}

\begin{equation} \label{eq:vel_sens_matrix}
    \boldsymbol{H}_{\omega} = \boldsymbol{I}
\end{equation}

Now constructing a block sensitivity matrix to span the full measurement model and state space, the full sensitivity matrix is shown in Equation \ref{eq:full_sens_matrix}.

\begin{equation} \label{eq:full_sens_matrix}
    \boldsymbol{H} = \begin{bmatrix}
        \boldsymbol{H}_{\alpha} & \boldsymbol{0} \\
        \boldsymbol{0} & \boldsymbol{H}_{\omega}
    \end{bmatrix} = \begin{bmatrix}
        \begin{bmatrix}
        \boldsymbol{ \left[ h_1 \times \right] } & \boldsymbol{ \left[ h_2 \times \right] } & \dots & \boldsymbol{ \left[ h_N \times \right] }
    \end{bmatrix}^T & \boldsymbol{0} \\
        \boldsymbol{0} & \boldsymbol{I}
    \end{bmatrix}
\end{equation}

The MATLAB Function within the Simulink model that is used to compute this is seen in Figure \ref{fig:sens_matrix_function}.

\begin{figure} [H]
    \centering
    \captionsetup{ justification = centering }
    \begin{lstlisting}
function H = constructH(h)

h_att = h(1:end-3);
% h_om = h(end-2:end);

n = uint32(length(h_att));

H_att = zeros([3, n]);

iters = n/3;

for i=1:iters
    H_att(:, (3*(i-1) + 1):(3*(i-1) + 1 + 2)) = vcross( h( (3*(i-1) + 1):(3*(i-1) + 1 + 2) ) ).';
end

H = [H_att.', zeros(size(H_att.'));
    zeros([3 3]), eye(3)];

    function V = vcross(v)
        V = [0, -v(3), v(2);
            v(3), 0, -v(1);
            -v(2), v(1), 0];
    end

end
    \end{lstlisting}
    \caption{Sensitivity Matrix Function}
    \label{fig:sens_matrix_function}
\end{figure}

\subsection{Problem 2 - Define and code your constant measurement error covariance matrix R which quantifies the uncertainty of your measurements. This can be picked as diagonal matrix with diagonal elements representing the variance of each parameter $\sigma^2_m$. Hint: If your instrument is well characterized you can use $\sigma_m$ applied in your simulation to generate the measurements from your ground-truth.}

A value for the noise covariance that was obtained via tuning and testing for better convergence is shown below.

\begin{equation*}
    \boldsymbol{R} = 2 \begin{bmatrix}
        1 & 0 & \dots & 0 \\
        0 & 1 &  & \vdots \\
        \vdots &  &  \ddots & \\
        0 & \dots & & 1
    \end{bmatrix} = 2 \boldsymbol{I}
\end{equation*}

\subsection{Problem 3 - Compute your modelled measurement vector at step k+1 from your state at step k+1. This transformation can be rigorous (non-linear, EKF) or approximate (linear, KF).}

The full concatenated measurement model is computed using both Equations \ref{eq:attitude_meas_model} and \ref{eq:vel_meas_model}. This full measurement model, seen in Equation \ref{eq:full_meas_model}, at this stage in simulation is supposed be be used at every time step. The model that computes this measurement model is seen in Figure \ref{fig:simulink_meas_model}.

\begin{equation} \label{eq:full_meas_model}
    \vec{h} = \begin{bmatrix}
        \vec{h}_{\alpha} \\ \vec{h}_{\omega}
    \end{bmatrix} = \begin{bmatrix}
        \boldsymbol{R}_{t+1 \vert t} \vec{v}_1 \\
        \boldsymbol{R}_{t+1 \vert t} \vec{v}_2 \\
        \vdots \\
        \boldsymbol{R}_{t+1 \vert t} \vec{v}_N \\
        \omega_x \\
        \omega_y \\
        \omega_z
    \end{bmatrix}
\end{equation}

\begin{figure}[H]
    \centering
    \captionsetup{ justification = centering }
    \includegraphics[width = 12cm]{}
    \caption{Simulink Model for Measurement Model Computation}
    \label{fig:simulink_meas_model}
\end{figure}

The assumption that this full model will be used at is not necessarily true for real space systems. This is due to varying availability of measurements from different sensors. Thus, a typical Kalman Filter on-board a spacecraft will handle measurements sequentially as they are made available. This functionality may be explored further in future work.

\subsection{Problem 4 - Compute your pre-fit residuals z by differencing modelled and actual measurement vector at k+1.}

The residual $\vec{z}$ is computed using Equation \ref{eq:meas_residual}.

\begin{equation} \label{eq:meas_residual}
    \vec{z} = \vec{y} - \vec{h}
\end{equation}

\subsection{Probelm 5 - The measurement update of the EKF needs H, P, R, and z to compute the Kalman gain K, the new estimated state and its associated covariance matrix.}

The Kalman gain is computed using Equation \ref{eq:kalman_gain}.

\begin{equation} \label{eq:kalman_gain}
    \boldsymbol{K}_t = \boldsymbol{\Sigma}_{t+1 \vert t} \boldsymbol{H}_t^T \left( \boldsymbol{H} \boldsymbol{\Sigma}_{t \vert t-1} \boldsymbol{H}_t^T + \boldsymbol{R}_t \right)^{-1}
\end{equation}

The measurement update for the state mean and covariance are computed using Equations \ref{eq:state_mean_update} and \ref{eq:state_covariance_update}.

\begin{equation} \label{eq:state_mean_update}
    \vec{x}_{t+1 \vert t+1} = \vec{x}_{t+1 \vert t} + \boldsymbol{K}_t \vec{z}_t
\end{equation}

\begin{equation} \label{eq:state_covariance_update}
    \boldsymbol{\Sigma}_{t+1 \vert t+1} = \boldsymbol{\Sigma}_{t+1 \vert t} - \boldsymbol{K}_t \boldsymbol{H}_t \boldsymbol{\Sigma}_{t+1 \vert t}
\end{equation}

The following MATLAB function, shown in Figure \ref{fig:meas_update_matlab_func}, uses the current sensitivity matrix, state covariance, measurement noise covariance, and the residual to compute an updated state mean and covariance estimate.

\begin{figure}
    \centering
    \captionsetup{ justification = centering}
    \begin{lstlisting}
function [mu_plus, Sig_plus] = measUpdate(mu_minus, Sig_minus, z, H, R_noise)

K = (Sig_minus * H.')/(H * Sig_minus * H.' + R_noise);

mu_plus = mu_minus + K * z;
Sig_plus = Sig_minus - K * H * Sig_minus;
    \end{lstlisting}
    \caption{Measurement Update Function}
    \label{fig:meas_update_matlab_func}
\end{figure}

\subsection{Problem 6 - Compute your post-fit residuals z by differencing modelled and actual measurement vector at k+1 using your new state. These should be smaller than the pre-fit residuals and should capture the standard deviation of your measurements at steady state.}

Following the measurement update, for a MEKF, the reset step must take place before the post fit residuals are computed. This update step follows Equation \ref{eq:quat_reset_step}.

\begin{equation} \label{eq:quat_reset_step}
    \vec{q}_{t+1 \vert t+1} = \begin{bmatrix}
        1 & -\alpha_1 & -\alpha_2 & -\alpha_3 \\
        \alpha_1 & 1 & \alpha_3 & -\alpha_2 \\
        \alpha_2 & -\alpha_3 & 1 & \alpha_1 \\
        \alpha_3 & \alpha_2 & -\alpha_1 & 1
    \end{bmatrix} \vec{q}_{t+1 \vert t}
\end{equation}

The model that computes this update step is shown in Figure \ref{fig:quat_reset_step}.

\begin{figure}[H]
    \centering
    \captionsetup{ justification = centering }
    \includegraphics{Images/PS8/}
    \begin{lstlisting}
function A = constructA(a)

ax = a(1);
ay = a(2);
az = a(3);

A = eye(4) + (1/2).*[0, -ax, -ay, -az;
                    ax, 0, az, -ay;
                    ay, -az, 0, ax;
                    az, ay, -ax, 0];
    \end{lstlisting}
    \caption{Simulink Model for Reset Step}
    \label{fig:quat_reset_step}
\end{figure}

Once this is computed, a rotation matrix can be obtained from the post-reset quaternion and used to recompute the measurement model. Now the residual can be computed using again Equation \ref{eq:meas_residual}.

For any plots and analysis regarding these residuals, the norm of this residual vector will be taken.

\subsection{Problem 7 - Produce plots showing true attitude estimation errors (estimate vs truth with statistics at steady state), formal or estimated attitude estimation errors (covariance from filter), pre- and post-fit residuals (with statistics at steady state), etc. Discuss the results, do they meet expectations? Is the true estimation error well described by the formal covariance? Are the measurements residuals consistent with the applied measurement errors? Hint: show estimation errors, do not overlap state estimate with reference truth.}

\subsection{Problem 8 - Start thinking/planning possible upgrades for the final project deliverable. Upgrades can go in several directions tailored to your project needs. For example, define different modes for attitude estimation using different sensors and algorithms based on your concept of operations. What would it take to implement a UKF instead of an EKF? Can you improve your dynamics and measurement models? Can you use measurements that are more representative of what your sensors are going to actually provide you? }