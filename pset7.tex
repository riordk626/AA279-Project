\section{\Large PROBLEM SET 7}

\subsection{Problem 1 - Introduce representative (from manufacturer or Wertz or first lectures) sensor errors in the form of constant bias and Gaussian noise with given standard deviation.} \label{sec:hw7problem1}

Bias and Gaussian noise were first introduced to the star tracker measurements through adding them to the unit vector that represented the location of an identified star in the ECI frame. Through analyzing star trackers produced by different manufacturers, the constant bias was set to 5 arcseconds and the mean of the Gaussian noise was set to 1 arcsecond based on the values given by Honeywell for their MST satellite. From there, the position vector of the star was converted to spherical coordinates so that the the noise and bias could be added to the angular components, $\theta$ and $\phi$. The bias was directly added to the unit vector, and the noise was added through multiplying the standard deviation by a normal distribution with a mean of zero and a standard deviation of one. The resultant simulink model and code for this are show below. 

\begin{figure}[H]
    \centering
    \captionsetup{ justification = centering }
    \includegraphics[width = 15cm]{Images/PS7/starTrackerErrorSimulink.png}
    \caption{Star Tracker Error Model in Simulink}
    \label{fig:starTrackerErrorSimulink}
\end{figure}

\begin{figure} [H]
    \centering
    \begin{lstlisting}
function rStarUnitPerturbed = perturbMeasurements(rStarUnitSamples, rStarUnitPerturbed)

% rStarUnitPerturbed = zeros(size(rStarUnitSamples));

% Bias and standard dev from known star tracker data (Honeywell MST)
mu = 5 * (pi / 180 / 3600); % Bias error in radians (converted from 5 arcseconds)
sigma = 1 * (pi / 180 / 3600); % Standard deviation of noise in radians (converted from 1 arcseconds)

nStars = length(rStarUnitSamples(1,:));

for i = 1:nStars

    % Get one star measurement
    rStarUnit = rStarUnitSamples(:,i);

    % Convert to spherical coordinates
    x = rStarUnit(1);
    y = rStarUnit(2); 
    z = rStarUnit(3);
    r = sqrt(x^2 + y^2 + z^2);
    theta = acos(z / r);
    phi = atan2(y, x);
    
    % Calculate the displacement due to bias error
    d_bias = mu;
    
    % Generate Gaussian noise and calculate displacement
    noise = sigma .* randn(1); % Angular noise in radians
    d_noise = noise;
    
    % Add the errors to the sperical coords
    theta_perturbed = theta + d_bias + d_noise;
    phi_perturbed = phi + d_bias + d_noise;
    
    % Convert back to spherical coordinates
    x_perturbed = r * sin(theta_perturbed) * cos(phi_perturbed);
    y_perturbed = r * sin(theta_perturbed) * sin(phi_perturbed);
    z_perturbed = r * cos(theta_perturbed);
    
    u_measured = [x_perturbed; y_perturbed; z_perturbed];
    
    % Normalize the measured vector to ensure it remains a unit vector
    u_measured = u_measured ./ norm(u_measured);

    rStarUnitPerturbed(:,i) = u_measured;
    
end
end
    \end{lstlisting}
    \caption{Star Tracker Noise}
    \label{fig:starTrackerNoise}
\end{figure}

For the magnetometer error, the unit vector of Earth's magnetic field was calculated and perturbed in the same way the unit vector for the star was in Figure \ref{fig:starTrackerNoise} and \ref{fig:starTrackerErrorSimulink}. The simulink model that built off of the previous model to perturb the star tracker is shown below as is the code used to calculated the unit vector of the magnetic field in the ECI frame.

\begin{figure}[H]
    \centering
    \captionsetup{ justification = centering }
    \includegraphics[width = 15cm]{Images/PS7/magnetometerErrorSimulink.png}
    \caption{Magnetometer Error Model in Simulink}
    \label{fig:magnetometerErrorSimulink}
\end{figure}

\begin{figure} [H]
    \centering
    \begin{lstlisting}
function B = magneticFieldPrinciple(r_orbit, mhat, Re, B0, R_ECItoP)

R = norm(r_orbit);
Rhat = r_orbit./R;

B = -(Re/R)^3 .* B0 .* (3*(dot(mhat, Rhat).*Rhat) - mhat);

B = B./norm(B);
end
    \end{lstlisting}
    \caption{Magnetometer Unit Vector Calculation}
    \label{fig:magnetometerNoise}
\end{figure}

For the $\omega$ error, the constant bias and the standard deviation of the Gaussian noise were once again pulled from a gyroscope manufacturer. This time they were pulled from Bosch Sensortec for the BMI160. The bias error was given as 3 $\degree$/$s$ and the standard deviation was set to  0.014 $\degree$/$\left(s \sqrt{Hz}\right)$. A bandwidth of 1 was assumed to convert the standard deviation to $\degree$/$s$. The standard deviation was then multiplied by a normal distribution with a mean of zero and a standard deviation of one. The resultant simulink model and code are shown below.

\begin{figure}[H]
    \centering
    \captionsetup{ justification = centering }
    \includegraphics[width = 15cm]{Images/PS7/GyroErrorSimulink.png}
    \caption{Gyroscope Error Model in Simulink}
    \label{fig:gyroErrorSimulink}
\end{figure}

\begin{figure} [H]
    \centering
    \begin{lstlisting}
function perturbedOmegaMeasured = OmegaMeasurementErrors(omegaMeasured)

% Convert degrees per second to radians per second for bias error
bias_error_deg_per_sec = 3; % Example bias error in degrees per second (from Bosch BMI160)
bias_error_rad_per_sec = bias_error_deg_per_sec * (pi / 180); % Convert to radians per second

% Convert noise density to standard deviation for a given bandwidth
noise_density_deg_per_sec_sqrtHz = 0.014; % Example noise density in degrees per second per sqrt(Hz)
bandwidth = 1; % Assuming a bandwidth of 1 Hz for simplicity
noise_std_dev_deg_per_sec = noise_density_deg_per_sec_sqrtHz * sqrt(bandwidth); % Standard deviation in degrees per second
noise_std_dev_rad_per_sec = noise_std_dev_deg_per_sec * (pi / 180); % Convert to radians per second

% Bias error (constant offset)
bias_error = bias_error_rad_per_sec .* ones(size(omegaMeasured));

% Generate Gaussian noise
noise = noise_std_dev_rad_per_sec .* randn(size(omegaMeasured));


% Add the errors to the unit vector
error = bias_error + noise;
perturbedOmegaMeasured = error + omegaMeasured;
end
    \end{lstlisting}
    \caption{Gyroscope Noise}
    \label{fig:gyroNoise}
\end{figure}

\subsection{Problem 2 - Re-apply the attitude determination algorithms from the previous pset. Plot attitude estimation error. Note that the attitude estimation error represents a rotation matrix (DCM) which quantifies how far the estimated
attitude is from the true attitude. You can use any parameterization to plot the attitude estimation errors corresponding to this DCM. Is the result consistent with the sensor bias and noise you have introduced?}

The error for the "undersampled" (2 available measurement) case from the previous assignment was once again plotted for the star tracker and magnetometer measurements. In order to see the error that was present in addition to the numerical integration error, the constant bias and Gaussian noise were scaled by a factor of 10 to make the distinction clear. 

\begin{figure}[H]
    \centering
    \captionsetup{ justification = centering }
    \includegraphics[width = 12cm]{Images/PS7/attitude_estimation_undersampled_det_default.png}
    \caption{Ground Truth vs. Estimated Attitude for Undersampled Deterministic Method with Feed Through Measurements with Noise and Bias}
    \label{fig:det_attitude_undersampled_default_noise}
\end{figure}

\begin{figure}[H]
    \centering
    \captionsetup{ justification = centering }
    \includegraphics[width = 12cm]{Images/PS7/attitude_estimation_undersampled_det_fictitious.png}
    \caption{Ground Truth vs. Estimated Attitude for Undersampled Deterministic Method with Fictitious Measurements with Noise and Bias}
    \label{fig:det_attitude_undersampled_fictitious_noise}
\end{figure}

\begin{figure}[H]
    \centering
    \captionsetup{ justification = centering }
    \includegraphics[width = 12cm]{Images/PS7/attitude_estimation_undersampled_q_default.png}
    \caption{Ground Truth vs. Estimated Attitude for Undersampled Statistical Method with Feed Through Measurements with Noise and Bias}
    \label{fig:stat_attitude_undersampled_default_noise}
\end{figure}

It can be seen in each of the plots that the added noise causes small oscillations in the measurements that aren't present in the ground truth values. These errors are small due to how accurate the star tracker measurements are but could still lead to issues if not properly accounted for. Additionally, there are some singularities where the error spikes due to the measurements of the magnetometer. These spikes could be caused by the magnetometer being weighted too heavily for its given accuracy or for an inaccuracy in the model. Future validation will be done to ensure that the spikes aren't caused by a model error.

For the $\omega$ measurements, Figure \ref{fig:omega_noise} was generated to show the error between the sensor and ground truth.

\begin{figure}[H]
    \centering
    \captionsetup{ justification = centering }
    \includegraphics[width = 12cm]{Images/PS7/obcVsGroundOmegas.png}
    \caption{Ground Truth vs. Estimated Angular Velocity with Noise}
    \label{fig:omega_noise}
\end{figure}

While the gaussian noise seems to dominate the error, the small difference from the constant bias can also be seen in $\omega_x$, $\omega_y$, and $\omega_z$. Even with the present noise, it can be seen that the values follow the same trends as the true values.

\subsection{Problem 3 - For small sensor errors, the DCM corresponds to a small rotation. Can you give an interpretation of small angles (e.g., in Euler angles and quaternions) to the obtained error DCM?}

As it can be seen in the plots from problem 2, the Gaussian noise results in small angle oscillations. Because of this, the the DCM matrix between the ground truth and measured values should be linearized and look like a small angle approximation matrix. That is, it should resemble the skews symmetric matrix with ones along the diagonal and small angles for the off diagonal values. To prove this, a DCM at one of the timesteps in the simulation is shown below.

\begin{align*}
\begin{bmatrix}
    -0.9890   &  0.1052  &  0.1036 \\
    0.0947  &  0.9903  & -0.1018 \\
   -0.1133 &  -0.0908 &  -0.9894
\end{bmatrix}
\end{align*}



\subsection{Problem 4 - Start modeling actual sensors in dedicated (Simulink or otherwise) subsystems which are part of the spacecraft. These models take inputs from ground-truth simulation and provides output measurements, including systematic and random errors. Take inspiration from overview of sensors discussed in class and
textbook for typical errors.}

Because Problem 1 already assumes that the sensors provide unit vector measurements that include error from a constant bias and Gaussian noise, a sufficient solution to this problem can already be found in Section \ref{sec:hw7problem1}. In the future, a lower level hardware model of the sensor may be implemented.

\subsection{Problem 5 - Designing and implement the time update of a KF/EKF to obtain the best estimate of the state from the available measurements and models:}

\subsubsection{Search in literature, define, and code a state transition matrix $\Phi$ which provides your state at step k+1 based on the state at step k. Verify that the output of this propagation step is consistent with the rigorous propagation of the attitude (numerical integration). Plot propagation errors as needed.}

For a Multiplicative Extended Kalman Filter (MEKF), the absolute attitude of the spacecraft is represented using a quaternion parameterization. This helps avoid singularities, while introducing the issue of a unitary norm constraint. To sidestep this issue, the state that is actually fed through the filter itself consists of a 3-component vector that represents a small deviation between a reference quaternion at the previous timestep and the reference quaternion for the current timestep. The expected value of these values is zero, and this component of the state is only altered during the measurement update. As a result, the time update step of MEKF will not update this value. The full state vector also contains the three angular velocity of the spacecraft about its principal axes. The full state vector can be written as seen in Equation \ref{eq:kf_state_def}.

\begin{equation} \label{eq:kf_state_def}
    \vec{x} = \begin{bmatrix}
        \alpha_1 \\ \alpha_2 \\ \alpha_3 \\ \omega_x \\ \omega_y \\ \omega_z
    \end{bmatrix}
\end{equation}

The discrete state transition matrix for this state in the time update is shown below in Equation \ref{eq:kf_statetrans_block}. For the discrete equations, the subscript $t$ denotes that the value corresponds to the time $t \in \left\{ 0, \Delta t, 2 \Delta t, ..., (N-1) \Delta t, N \Delta t \right\}$.

\begin{equation} \label{eq:kf_statetrans_block}
    \boldsymbol{\Phi}_t = \begin{bmatrix}
        \boldsymbol{\Phi}_{\alpha,t} & \boldsymbol{0} \\
        \boldsymbol{0} & \boldsymbol{\Phi}_{\omega,t}
    \end{bmatrix}
\end{equation}

Where $\boldsymbol{\Phi}_{\alpha,t}$ and $\boldsymbol{\Phi}_{\omega,t}$ are defined in Equations \ref{eq:alpha_STM} and \ref{eq:omega_STM}.

\begin{equation} \label{eq:alpha_STM}
    \boldsymbol{\Phi}_{\alpha,t} = \boldsymbol{I} + s_{\omega} \boldsymbol{\left[ \vec{\omega} \times \right]} + \left( \frac{1 - c_{\omega}}{\omega ^2} \right) \boldsymbol{\left[ \vec{\omega} \times \right]} \boldsymbol{\left[ \vec{\omega} \times \right]}
\end{equation}

\begin{equation} \label{eq:omega_STM}
    \boldsymbol{\Phi}_{\omega,t} = \boldsymbol{I} + \begin{bmatrix}
        0 & \frac{I_y - I_z}{I_x} \omega_{z,t} & \frac{I_y - I_z}{I_x} \omega_{y,t} \\
        \frac{I_z - I_x}{I_y} \omega_{z,t} & 0 & \frac{I_z - I_x}{I_y} \omega_{x,t} \\
        \frac{I_x - I_y}{I_z} \omega_{y,t} & \frac{I_x - I_y}{I_z} \omega_{x,t} & 0
    \end{bmatrix} \frac{\Delta t}{2}
\end{equation}

Where

\begin{equation*}
    \omega \triangleq \Vert \vec{\omega} \Vert
\end{equation*}
\begin{equation*}
    c_{\omega} \triangleq \cos{\frac{\omega \Delta t}{2}}
\end{equation*}
\begin{equation*}
    s_{\omega} \triangleq \frac{1}{\omega} \sin{\frac{\omega \Delta t}{2}}
\end{equation*}

Therefore the discrete time update step for the state can be seen in Equation \ref{eq:kf_state_time_update}.

\begin{equation} \label{eq:kf_state_time_update}
    \vec{x}_{t + 1} = \boldsymbol{\Phi}_t \vec{x}_{t + 1} + \boldsymbol{B}_t \vec{u}_t
\end{equation}

The matrix $\boldsymbol{B}_t$ and $\vec{u}_t$ will be defined later.

\subsubsection{Search in literature, define, and code a control input matrix B which provides the increment to your state at step k+1 due to a control torque at step k. Hint: optional at this stage since you do not have a controller yet.}

In Equation \ref{eq:kf_state_time_update}, the values $\boldsymbol{B}_t$ and $\vec{u}_t$ were introduced. For this choice of state, the matrix $\boldsymbol{B}_t$ is defined below.

\begin{equation*}
    \boldsymbol{B}_t = \begin{bmatrix}
        \boldsymbol{0} \\
        \boldsymbol{B}_{\omega,t}
    \end{bmatrix}
\end{equation*}

With the block $\boldsymbol{B}_{\omega,t}$ being defined as follows.

\begin{equation*}
    \boldsymbol{B}_t = \begin{bmatrix}
        1/I_x & 0 & 0 \\
        0 & 1/I_y & 0 \\
        0 & 0 & 1/I_z
    \end{bmatrix} \Delta t
\end{equation*}

The vector $\vec{u}_t$ represents the control inputs, which are the moments applied along the three principal axes. Therefore the vector can be defined as below.

\begin{equation*}
    \vec{u}_t = \begin{bmatrix}
        M_{x,t} \\ M_{y,t} \\ M_{z,t}
    \end{bmatrix}
\end{equation*}

\subsubsection{(a) and (b) allow you to propagate the state from k to k+1 including the known control input torques. Hint: Initially you will design the filter by neglecting any control torque from your simulation.}

\subsubsection{Define and code an initial state error covariance matrix P which quantifies the uncertainty of your initial state. This can be picked as diagonal matrix with diagonal elements representing the variance of each state parameter $\sigma ^2$. Initially you can neglect cross-covariance terms assuming that errors of various state components are not correlated.}

For the initial covariance, denoted as $\boldsymbol{\Sigma}_{0 \vert 0}$, a diagonal matrix with 6 elements was generated with an order of magnitude that corresponds with the error seen in the simulation results above that compare the ground truth to the determined attitude.

\begin{equation*}
    \boldsymbol{\Sigma}_{0 \vert 0} = 10^{-3} \begin{bmatrix}
        1 & 0 & 0 & 0 & 0 & 0 \\
        0 & 1 & 0 & 0 & 0 & 0 \\
        0 & 0 & 1 & 0 & 0 & 0 \\
        0 & 0 & 0 & 1 & 0 & 0 \\
        0 & 0 & 0 & 0 & 1 & 0 \\
        0 & 0 & 0 & 0 & 0 & 1 \\
    \end{bmatrix}
\end{equation*}

\subsubsection{The time update of the EKF needs $\Phi$, B, and P. Hint: You could increment your navigation performance by keeping the filter receptive to new measurements at steady state through the addition of constant process noise Q at each step. Initially you can define Q similar to P but much smaller (e.g., 1/10 or 1/100).}

The time update for the covariance follows Equation \ref{eq:kf_cov_time_update}.

\begin{equation} \label{eq:kf_cov_time_update}
    \boldsymbol{\Sigma}_{t + 1 \vert t} = \boldsymbol{\Phi}_t \boldsymbol{\Sigma}_{t \vert t} \boldsymbol{\Phi}_t^T + \boldsymbol{Q}_t
\end{equation}

The matrix $\boldsymbol{Q}_t$ represents the process noise. This was chosen heuristically to be equivalent to the intiial covariance scaled by a factor of $1/100$. 

\begin{equation*}
    \boldsymbol{Q}_t = \frac{1}{100} \boldsymbol{\Sigma}_{0 \vert 0} = 10^{-5} \begin{bmatrix}
        1 & 0 & 0 & 0 & 0 & 0 \\
        0 & 1 & 0 & 0 & 0 & 0 \\
        0 & 0 & 1 & 0 & 0 & 0 \\
        0 & 0 & 0 & 1 & 0 & 0 \\
        0 & 0 & 0 & 0 & 1 & 0 \\
        0 & 0 & 0 & 0 & 0 & 1 \\
    \end{bmatrix}
\end{equation*}

\subsection{Problem 6 - Produce plots showing true attitude estimation errors (estimate vs truth with statistics), formal or estimated attitude estimation errors (covariance from filter). Discuss the results, do they meet expectations? How well is the true estimation error described by the formal covariance? Note that we are only implementing the time update even if we call them “attitude estimates and estimation errors”.}

Figure \ref{fig:kf_quat_prop} shows the ground truth attitude represented as Euler angles, the estimated attitude (converted from the propagated reference quaternion to Euler angles), and the error between the two.

\begin{figure}[H]
    \centering
    \captionsetup{ justification = centering}
    \includegraphics[width = 15cm]{Images/PS7/kalman_filter_time_update_error_attitude.png}
    \caption{Ground Truth vs. Kalman Filter Time Update Estimate for Attitude}
    \label{fig:kf_quat_prop}
\end{figure}

Figure \ref{fig:kf_omega_prop} shows the angular velocity portion of the state compared with the ground truth angular velocity. The attitude error states were omitted for this step as their evolution is trivial.

\begin{figure}[H]
    \centering
    \captionsetup{ justification = centering}
    \includegraphics[width = 15cm]{Images/PS7/kalman_filter_time_update_error_velocities.png}
    \caption{Ground Truth vs. Kalman Filter Time Update Estimate for Angular Velocities}
    \label{fig:kf_omega_prop}
\end{figure}

In both of these figures, a delay can be seen at the beginning of the sim before the first update step is performed to match the ground truth and then linearly propagate the state forward. Beyond this point the errors between the ground truth and the time update estimate are quite small. Singularities still exist that cause spikes in the   error, but this is to be expected.

In Figure \ref{fig:kf_time_update_attitude_statistics} the bounds corresponding to the variance on each attitude error state is shown, while in Figure \ref{fig:kf_time_update_vel_statistics} the variance is shown for each component of angular velocity.

\begin{figure}[H]
    \centering
    \captionsetup{ justification = centering}
    \includegraphics[width = 15cm]{Images/PS7/kalman_filter_att_cov_bounds.png}
    \caption{1-$\sigma$ Bounds on Attitude Error State}
    \label{fig:kf_time_update_attitude_statistics}
\end{figure}

\begin{figure}[H]
    \centering
    \captionsetup{ justification = centering}
    \includegraphics[width = 15cm]{Images/PS7/kalman_filter_omega_cov_bounds.png}
    \caption{1-$\sigma$ Bounds on Angular Velocity State}
    \label{fig:kf_time_update_vel_statistics}
\end{figure}

Both currently grow indefinitley, as there is no measurement update to reduce uncertainty in the state estimate.

The Simulink model used to implement the Kalman Filter is shown below in Figure \ref{fig:kf_simulink_model}.

\begin{figure}[H]
    \centering
    \captionsetup{ justification = centering }
    \includegraphics[trim={0cm 5cm 0cm 4cm},clip,width = 15cm]{Images/PS7/kalmanFilter-01.png}
\end{figure}

\begin{figure}[H]
    \centering
    \captionsetup{ justification = centering }
    \includegraphics[trim={0cm 2cm 0cm 3cm},clip,width = 15cm]{Images/PS7/kalmanFilter-04.png}
\end{figure}

\begin{figure}[H]
    \centering
    \captionsetup{ justification = centering }
    \includegraphics[trim={0cm 18cm 15cm 0cm},clip,width = 15cm]{Images/PS7/kalmanFilter-05.png}
\end{figure}

\begin{figure}[H]
    \centering
    \captionsetup{ justification = centering }
    \includegraphics[trim={0cm 18cm 15cm 0cm},clip,width = 15cm]{Images/PS7/kalmanFilter-06.png}
\end{figure}

\begin{figure}[H]
    \centering
    \captionsetup{ justification = centering }
    \includegraphics[trim={0cm 18cm 15cm 0cm},clip,width = 15cm]{Images/PS7/kalmanFilter-07.png}
\end{figure}

\begin{figure}[H]
    \centering
    \captionsetup{ justification = centering }
    \includegraphics[trim={0cm 9cm 15cm 0cm},clip,width = 15cm]{Images/PS7/kalmanFilter-08.png}
\end{figure}

\begin{figure}[H]
    \centering
    \captionsetup{ justification = centering }
    \includegraphics[trim={0cm 16cm 15cm 0cm},clip,width = 15cm]{Images/PS7/kalmanFilter-09.png}
    \caption{Multiplicative Kalman Filter Model}
    \label{fig:kf_model}
\end{figure}